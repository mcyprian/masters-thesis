%%%%%%%%%%%%%%%%%%%%%%%%%%%%%%%%%%%%%%%%%%%%%%%%%%%%%%%%%%%%%%%%%%%%
%% I, the copyright holder of this work, release this work into the
%% public domain. This applies worldwide. In some countries this may
%% not be legally possible; if so: I grant anyone the right to use
%% this work for any purpose, without any conditions, unless such
%% conditions are required by law.
%%%%%%%%%%%%%%%%%%%%%%%%%%%%%%%%%%%%%%%%%%%%%%%%%%%%%%%%%%%%%%%%%%%%

\documentclass[
  digital, %% This option enables the default options for the
           %% digital version of a document. Replace with `printed`
           %% to enable the default options for the printed version
           %% of a document.
  twoside, %% This option enables double-sided typesetting. Use at
           %% least 120 g/m² paper to prevent show-through. Replace
           %% with `oneside` to use one-sided typesetting; use only
           %% if you don’t have access to a double-sided printer,
           %% or if one-sided typesetting is a formal requirement
           %% at your faculty.
  table,   %% This option causes the coloring of tables. Replace
           %% with `notable` to restore plain LaTeX tables.
  lof,     %% This option prints the List of Figures. Replace with
           %% `nolof` to hide the List of Figures.
  lot,     %% This option prints the List of Tables. Replace with
           %% `nolot` to hide the List of Tables.
  %% More options are listed in the user guide at
  %% <http://mirrors.ctan.org/macros/latex/contrib/fithesis/guide/mu/fi.pdf>.
]{fithesis3}
%% The following section sets up the locales used in the thesis.
\usepackage[resetfonts]{cmap} %% We need to load the T2A font encoding
\usepackage[T1,T2A]{fontenc}  %% to use the Cyrillic fonts with Russian texts.
\usepackage[
  main=english, %% By using `czech` or `slovak` as the main locale
                %% instead of `english`, you can typeset the thesis
                %% in either Czech or Slovak, respectively.
  english, german, russian, czech, slovak %% The additional keys allow
]{babel}        %% foreign texts to be typeset as follows:
%%
%%   \begin{otherlanguage}{german}  ... \end{otherlanguage}
%%   \begin{otherlanguage}{russian} ... \end{otherlanguage}
%%   \begin{otherlanguage}{czech}   ... \end{otherlanguage}
%%   \begin{otherlanguage}{slovak}  ... \end{otherlanguage}
%%
%% For non-Latin scripts, it may be necessary to load additional
%% fonts:
\usepackage{paratype}
\def\textrussian#1{{\usefont{T2A}{PTSerif-TLF}{m}{rm}#1}}
%%
%% The following section sets up the metadata of the thesis.
\thesissetup{
    date          = \the\year/\the\month/\the\day,
    university    = mu,
    faculty       = fi,
    type          = mgr,
    author        = Michal Cyprian,
    gender        = m,
    advisor       = Jan Horák,
    title         = {High-availability for PostgreSQL in OpenShift},
    %% TeXtitle      = {The Proof of $\mathsf{P}=\mathsf{NP}$},
    keywords      = {PostgreSQL, OpenShift, Kubernetes, High-availability, Replication, Operator SDK},
    %% TeXkeywords   = {keyword1, keyword2, \ldots},
    abstract      = {%
      This is the abstract of my thesis, which can

      span multiple paragraphs.
    },
    thanks        = {%
      These are the acknowledgements for my thesis, which can

      span multiple paragraphs.
    },
    bib           = example.bib,
}
\usepackage{makeidx}      %% The `makeidx` package contains
\makeindex                %% helper commands for index typesetting.
%% These additional packages are used within the document:
\usepackage{paralist} %% Compact list environments
\usepackage{amsmath}  %% Mathematics
\usepackage{amsthm}
\usepackage{amsfonts}
\usepackage{url}      %% Hyperlinks
\usepackage{markdown} %% Lightweight markup
\usepackage{tabularx} %% Tables
\usepackage{tabu}
\usepackage{booktabs}
\usepackage{listings} %% Source code highlighting
\lstset{
  basicstyle      = \ttfamily,
  identifierstyle = \color{black},
  keywordstyle    = \color{blue},
  keywordstyle    = {[2]\color{cyan}},
  keywordstyle    = {[3]\color{olive}},
  stringstyle     = \color{teal},
  commentstyle    = \itshape\color{magenta},
  breaklines      = true,
}
\usepackage{floatrow} %% Putting captions above tables
\floatsetup[table]{capposition=top}
%% The following code fixes the rendering of BibLaTeX ISO 690
%% references in old TeX Live (such as the one at Overleaf).
\thesisload
\makeatletter
\def\thesis@biblatexiso@fix@package{iso-numeric.bbx}
\def\thesis@biblatexiso@fix@end{\relax}
\newif\ifthesis@biblatexiso@fix@
\thesis@biblatexiso@fix@false
\def\thesis@biblatexiso@fix@next#1,{%
  \def\thesis@biblatexiso@fix@current{#1}%
  \ifx\thesis@biblatexiso@fix@current\thesis@biblatexiso@fix@package
    \thesis@biblatexiso@fix@true
  \fi
  \ifx\thesis@biblatexiso@fix@current\thesis@biblatexiso@fix@end
    \expandafter
    \@gobble
  \fi
  \thesis@biblatexiso@fix@next
}
\expandafter\expandafter\expandafter\thesis@biblatexiso@fix@next\@filelist,\relax,
\ifthesis@biblatexiso@fix@
  \defbibenvironment{bibliography}
    {\list%
       {\MethodFormat}%
       {\setlength{\labelwidth}{\labelnumberwidth}%
        \setlength{\leftmargin}{\labelwidth}%
        \setlength{\labelsep}{\biblabelsep}%
        \addtolength{\leftmargin}{\labelsep}%
        \setlength{\itemsep}{\bibitemsep}%
        \setlength{\parsep}{\bibparsep}}%
        \renewcommand*{\makelabel}[1]{\hss##1}
        }%
    {\endlist}%
  {\item}%
\fi
\makeatother
\begin{document}
\chapter*{Introduction}
\addcontentsline{toc}{chapter}{Introduction}

We live in the computer age. People can access information and knowledge faster and easier than ever before. The evolution of information technology in the last few decades have a great impact on the society. The heavy personal computers have been replaced by tiny devices, laptops, cell phones and smart watches. The performance have been increased, internet connection is available almost everywhere and the number of easy to use applications and web pages for various purposes and have grown significantly.

This revolution also brought a lots of new challenges and rapid changes on the other side of computer industry. All the data centers, where pieces of information are stored, computers, which runs those services, web pages and applications must have been adapted to the requirements of this age equally well.

The main goals of web service providers are to make the service high available, capable of handling big number of requests each second and update running services smoothly. The service should survive disk corruption on some of the computers it is running on, small network issues and sometimes event natural disasters in the area, where data center is located. Ideally, the user shouldn't experience any downtime of the service, when it's being updated to newer version. Software engineers have started an effort to develop complex platforms able to face such challenges in the automated way.

Automated platforms currently work quite well for a simple applications. Additional level of complexity is added when systems storing a specific state, such as databases are managed automatically by the platform. This area is still relatively unexplored and provides a plenty of fascinating design problems to be solved. That is the main reason why I have chosen integrating of high available database setup into on of the fully automated platform as a topic for my thesis.

The main goal of this thesis is to explore an existing tools and solutions for both, database high availability and the automated platforms, choose the most suitable tools to be integrated into a single system and implement the missing automation logic. The resulting system should be able to handle real world use cases of high available database autonomously. The final part of the thesis will be to check and evaluate behaviour of the system in different scenarios.

TODO: Navod na citanie prace, struktura prace v kapitole XXX je popsano...

\chapter{Container orchestration}
The new approach to design, create and deploy applications is much different from the traditional way, which includes many manual stages. The main goals are to get new services to the market faster, minimize the cost and increase scalability of the services. Automation of all the stages of service life cycle, using principles like miroservices architecture, countinuous delivery and container orchestration platforms is the key to achieve the goals. This particular approach to application life cycle combined with new tooling is usually called "Cloud Native" \cite{cloud_native}.

Following chapter describes concrete container orchestration tools that are the core components of Cloud Native approach.

\section{Linux containers}
The idea of isolating a single process from the rest of Unix-like system has been around for decades. There are multiple reasons for isolating part of the system. Sandboxing non trusted application can prevent security threats. Installing application dependencies into a separate file system can help to avoid requirement version conflicts and isolated program alongside it's dependencies can be distributed to different machines with lower risk of introducing issues.

The concept of chroot \cite{unix_chroot} was introduced during the development of Unix Version 7 in 1979. The chroot system call serves to isolate the process by changing the root directory of the new process to the different path in the file system. The process isolated by chroot and its children can only access the directories within its own directory tree. This concept is considered to be the first implementation of light weight isolation. The chroot system call has still its use cases forty years after its original implementation.

The concept of chroot improved in several ways served as a base for FreeBSD Jails \cite{freebsd_jails}. The separation implemented by jails is not limited to the file system access. All the system resources like users, processes and networking subsystem. The administrator can divide the system into several independent units. Each jail can have an IP address and its own configuration.

In 2006 Google engineers launched project called Process Containers. This feature allows partitioning of resources such as CPU time, system memory, disk and network bandwith into groups and assigning tasks to these groups. The resource limits for a collection of processes can be specified this way. This feature was later renamed to Control Groups (cgroups) and merged into Linux kernel 2.6.24 \cite{cgroups}.

Another important concept in Linux process isolation are the Linux Namespaces \cite{namespaces}, which provide processes with their own view of the system resources. There are multiple namespaces managing visibility of different kinds of resources. The processes within a specific PID namespace can only see processes in the same namespace. Similar principle applies to the network, mount, user and other namespaces.
Control Groups alongside namespaces are a fundamental aspects of an operating system level virtualization on Linux. Isolated virtual instances using a single shared Linux kernel are called containers.

LXC, Docker, rkt, CRI-O and many other are the implementations of Linux container runtime, based on the similar principles which have appeared since 2008. Not only container runtimes but also other kinds of tools have been developed. After the initial period, when the technology was relatively unstable and challenging, it finally become mature enough to be widely used in production.

When comparing Linux containers to the concept of virtual machines, virtualization runs an entire guest operating system in a virtual machine while container share kernel of the host. The result is that containerization provides shorter build and setup times, smaller image size and real-time provisioning and scalability. In the other hand process level isolation is less secure than full isolation of the host and guests systems.
In general Linux containers satisfies the requirements of Cloud Native approach workflows better than heavyweight virtualization methods.

\section{Kubernetes}
\subsection{Containerized applications management}
\subsection{OpenShift}
\subsection{Operators}

\chapter{PostgreSQL database}
\section{Object-relational database system}
\section{Database replication}
\subsection{Replication tools}
\subsection{Replication scenarios}

\chapter{PostgreSQL Operator}
\section{Operator design}
\subsection{Operator API}
\section{Cluster initialization}
\section{Node interaction}
\section{Testing}

\appendix %% Start the appendices.
\chapter{An appendix}
Here you can insert the appendices of your thesis.

\end{document}
